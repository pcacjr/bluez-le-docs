\documentclass[11pt]{article}
\usepackage[utf8]{inputenc}
\usepackage{graphicx}
\usepackage{fancyhdr}

\pagestyle{fancy}
\lhead{}
\chead{}
\rhead{}
\fancyfoot[C] {This document is confidential}
\fancyfoot[R] {\thepage}
\fancyhead[C] {BlueZ BTLE How-To}
\fancyhead[R] {\thepage}

\setlength{\parskip}{10pt}

\title{BlueZ BTLE How-To}
\author{Claudio Takahasi \\ claudio.takahasi@openbossa.org \and
        João Paulo Rechi Vita \\ jprvita@openbossa.org \and
        Paulo Alcantara \\ paulo.alcantara@openbossa.org}

\begin{document}

\maketitle
\newpage

\tableofcontents
\newpage

\section{Introduction}

This document describes how to test and use Bluetooth Low-Energy on Linux
systems, using the most recent implementation available at the moment of this
writing. The instructions are based on a Ubuntu Linux 11.10, since its the most
widely deployed Linux distribution nowadays. Nonetheless, the process for other
distributions should be very similar.

\section{Dependencies}

Bellow are listed the libraries and tools needed to compile BlueZ for BTLE
development. The list uses the packages name on Ubuntu 11.10.

\begin{itemize}
 \item git
 \item automake
 \item byacc
 \item flex
 \item libtool
 \item libdbus-1-dev
 \item libglib2.0-dev
 \item libcap-ng-dev
 \item libusb-dev
 \item libudev-dev
 \item libreadline-dev
\end{itemize}

To install packages on Ubuntu, use \verb|apt-get|, like shown bellow.

\begin{verbatim}
$ sudo apt-get install <package>
\end{verbatim}

Make sure you have {\em libreadline-dev} installed, otherwise \verb|gatttoll| is not
built.

\section{Getting the source code}

\subsection{Linux kernel}

BTLE development is focused on the new Management API, which requires a very
recent kernel tree. The most up-to-date kernel tree with several LE-specific
fixes can be found on the {\em integration} branch of the following tree on
GitHub. Eventually this branch will be integrated on the official kernel trees.
Don't bother trying to compile other branches, since they have experimental
changes. To access this code, clone the tree and checkout the branch:

Additionally, one should take special care when doing changes on top of this
branch, since it's going to be constantly rebased on top of the {\em
bluetooth-next} kernel tree.

\begin{verbatim}
$ git clone git://github.com/aguedes/linux.git
$ cd linux
$ git checkout integration
\end{verbatim}

\subsection{Userspace}

The upstream BlueZ tree has all the code necessary for testing and do further
BTLE development on Linux.

\begin{verbatim}
$ git clone git://git.kernel.org/pub/scm/bluetooth/bluez.git
\end{verbatim}

\section{Building and running}

\subsection{Linux kernel}

Now, copy the Linux kernel configuration file from Ubuntu as \verb|.config| to
the Linux kernel sources directory, and generate a new configuration based on
the Ubuntu one.

\begin{verbatim}
$ cd linux
$ cp /boot/config-`uname -r` .config
$ make oldconfig
\end{verbatim}

You should just press Enter for each question until this command returns. Now
you're ready to compile your kernel. To build and generate a Debian package
that can be installed on Ubuntu the packages {\em fakeroot} and {\em dpkg-dev}
are needed.  Make sure they're installed and run:

\begin{verbatim}
$ make deb-pkg
\end{verbatim}

Then install the new kernel packages with {\em dpkg}:

\begin{verbatim}
$ cd ..
$ sudo dpkg -i linux-image* linux-headers*
\end{verbatim}

After the installation is complete, reboot to use the new kernel. To enable the
new MGMT interface a parameter should passed to the bluetooth kernel module.
Another paramenter is also needed to enable LE discovery.  After unloading all
bluetooth adapters ({\em hci0} to {\em hciN}) and bluetooth kernel modules, the
{\em bluetooth} module should be reloaded with {\em enable\_mgmt=1} and {\em
enable\_le=1} (make sure substitute {\em hciX} with the name of the adapter,
for all bluetooth adapters on the system).

\begin{verbatim}
$ sudo hciconfig hciX down
$ sudo rmmod rfcomm bnep btusb bluetooth
$ sudo modprobe bluetooth enable_mgmt=1 enable_le=1
\end{verbatim}

\subsection{Userspace}

BlueZ has a script that does the {\em bootstrap} and {\em configure} phases
with the recommended options for developers. Using it will ease the test and
development processes and is highly recommended. After downloading the source
code, run this script to prepare for compilation.

\begin{verbatim}
$ cd bluez
$ ./bootstrap-configure
\end{verbatim}

Due to a bug in Debian (thus Ubuntu) packaging, if you have the {\em check}
package installed you have to pass the link flag for the correct version of
{\em libcheck}.

\begin{verbatim}
CHECK_LIBS="-lcheck_pic" ./bootstrap-configure
\end{verbatim}

And then build the binaries with {\em make}.

\begin{verbatim}
$ make
\end{verbatim}

Now you're ready to run the Bluetooth daemon you've just compiled. First stop
the one that came with you distro:

\begin{verbatim}
$ sudo service bluetooth stop
\end{verbatim}

And then manually start it from the source tree, without forking into a daemon
(n) and with debug enabled (d).

\begin{verbatim}
$ sudo src/bluetoothd -nd
\end{verbatim}

\subsection{Reporting errors}
FIXME: Add here instructions to report errors: hcidump logs, kernel logs, bluetoothd traces, valgrind

\section{Testing Management Interface}

Bluetooth Management Interface is a replacement for the traditional HCI
interface, it is a new kernel/userspace API. It aims to move some core host
features to the kernel hiding from the userspace some hardware/controller
specific features. It will NOT be a 1 to 1 HCI command mapping, the idea is
to group a set o HCI commands in one management command.

Management Interface usage is transparent to the userspace. BlueZ supports
an abstraction to hide from the user the active interface(aka adapter ops
plugin). There is one plugin for management(with higher priority) and another
plugin for HCI, if Management Interface is not enabled in the Kernel, the
fallback is the HCI plugin.

To check if MGMT interface and LE discovery are enabled, the following commands
should return a \verb|Y|.

\begin{verbatim}
$ cat /sys/module/bluetooth/parameters/enable_mgmt
Y
$ cat /sys/module/bluetooth/parameters/enable_le
Y
\end{verbatim}

\subsection{btmgmt}
describe here how to pair and start discovery using btmgmt


\section{Enabling debug}

howto enable kernel debug: dmesg debug level, dynamic debug

\subsection{Enabling kernel debug}

debugfs is a virtual file system that allows developers use to put debugging files into.

\begin{enumerate}
	\item Check if dynamic debug and debug FS are enabled in the kernel settings(.config):

	\begin{verbatim}
		CONFIG_DEBUG_FS=y
		CONFIG_DYNAMIC_DEBUG=y
	\end{verbatim}

	\item Tracking bluetooth files logs

	\begin{verbatim}
	Dynamic debugging is controlled via the "dynamic_debug/control" file.
	Add the bluetooth files that you wanna track the dynamic printk
	messages in the control file.

		$ echo "file net/bluetooth/smp.c +pf\" >> /sys/kernel/debug/dynamic_debug/control
		$ echo "file net/bluetooth/mgmt.c +pf\" >> /sys/kernel/debug/dynamic_debug/control
		$ echo "file net/bluetooth/hci_sock.c +pf\" >> /sys/kernel/debug/dynamic_debug/control
		$ echo "file net/bluetooth/hci_event.c +pf\" >> /sys/kernel/debug/dynamic_debug/control
		$ echo "file net/bluetooth/hci_core.c +pf\" >> /sys/kernel/debug/dynamic_debug/control
		$ echo "file net/bluetooth/hci_conn.c +pf\" >> /sys/kernel/debug/dynamic_debug/control
		$ echo "file net/bluetooth/l2cap_sock.c +pf\" >> /sys/kernel/debug/dynamic_debug/control
		$ echo "file net/bluetooth/l2cap_core.c +pf\" >> /sys/kernel/debug/dynamic_debug/control
	\end{verbatim}

	\item Setting level of messages printed to console

		\begin{verbatim}
		$ sudo dmesg --console-level debug
		\end{verbatim}
\end{enumerate}

\section{Testing using Python scripts}

BlueZ(bluetoothd) exposes a D-Bus API to allow the applications to configure
Bluetooth adapters, manage discovery, {\em create} device and interact with
services.

D-Bus(http://http://dbus.freedesktop.org/) is widely used in Linux
distributions as interprocess communication mechanism. Python offers
language bindings for D-Bus, exposing a simple syntax to create 
{\em proxy objects} to interact with remote applications.

For BlueZ, the device {\em object} represents the remote Bluetooth device.
CreateDevice or CreatePairedDevice is the starting point to access
remote device's services. Bluetooth Low Energy support in BlueZ aligns
BR/EDR and Low Energy, device discovery and device {\em create} operations
share the same API.

\subsection{Discovering Devices}

Interleaved discovery can be executed through management interface or the
standard HCI interface. The difference is where the code is implemented:
for Management, the discovery is in the kernel instead of userspace
plugin(HCI adapter ops plugin).

When running with Management Interface, {\em enable\_le} module parameter
needs to be enabled to execute interleaved discovery, otherwise only
inquiry will be executed.

Start Discovery command in Management Interface has a bit-wise parameter to
inform which remote address type is expected. At the moment, interleaved
discovery is the default method hard-coded in bluetoothd when Management
Interface is in use. btmgmt tool can be used instead to test discovery
with address type filter. The following values are allowed:

\begin{verbatim}
	1   BR/EDR
	6   LE (public & random)
	7   BR/EDR/LE (interleaved discovery)
\end{verbatim}

Testing discovery through D-Bus:
\begin{verbatim}
	$ ./test-discovery
\end{verbatim}

Testing discovery directly through Management socket:
\begin{verbatim}
	Finding Low Energy Devices: Public and Random Addresses:
	$ ./btmgmt find -l

	Finding BR/EDR devices:
	$ ./btmgmt find -b
\end{verbatim}
Default is interleaved discovery: 5.12s of LE scanning followed by 5.12s
of inquiry.

\subsection{Managing Devices}

The simple-agent is a Python script for testing purposes only and which is part
of the BlueZ toolchain. This script is used for when pairing Bluetooth
LE devices. If you want to pair with a Bluetooth LE device without using the
gnome-bluez package, for example, then this is right tool for that.

Registering a global agent to handle incoming pairing requests:
\begin{verbatim}
  $ simple-agent
\end{verbatim}

If it all works, you should get the message ``Agent registered'' on
that console. You can now pairing from your device, and the
script will ask you for the passcode on this console, you type it and
confirm with enter, that's all. You can now also shut down the agent
using the key sequences Ctrl + c (that sends a SIGINT signal from
the console to the running program itself).

If you wanna pair with a given device only:
\begin{verbatim}
  $ simple-agent hciX <Bluetooth Address>
\end{verbatim}

-c or --capability option can be used to change the default {\em
KeyboardDisplay} value. Other possible values are: {\em DisplayOnly}, {\em
DisplayYesNo} {\em KeyboardOnly}, {\em NoInputNoOutput}

After {\em creating}a new device, test-device script can be used to interact
with the remote device.

There are basically 11 operations that you can perform with test-device script:
\begin{enumerate}
  \item List all known Bluetooth devices.
    \begin{verbatim}
      $ test-device list
    \end{verbatim}
  \item List GATT services from a given device.
    \begin{verbatim}
      $ test-device services <Bluetooth address>
    \end{verbatim}
  \item Create a remote device from a given Bluetooth address. Triggers the
	  SDP or discover all primary services.
    \begin{verbatim}
      $ test-device create <Bluetooth address>
    \end{verbatim}
  \item Remove a device from either a Bluetooth address or a path in
        D-Bus's object path form.
    \begin{verbatim}
      $ test-device remove <Bluetooth address or Service object path>
    \end{verbatim}
  \item Disconnect a remote device.
    \begin{verbatim}
      $ test-device disconnect <Bluetooth address>
    \end{verbatim}
  \item Discover a device from a given Bluetooth address and also an
        optional pattern string. If this option argument is the string
        ``yes'', for example, then the DiscoverDevices() method of the
        org.bluez.Device interface will be called and find presumably
        the device that matches this pattern. Method used to update the
	service list using SDP or discover all primary services.
    \begin{verbatim}
      $ test-device discover <Bluetooth address> [pattern]
    \end{verbatim}
  \item Return the device class (retrieved from
        org.bluez.Device.GetProperties() method which returns all the
        properties of the attached Bluetooth adapter) from a given
        Bluetooth address.
    \begin{verbatim}
      $ test-device class <Bluetooth address>
    \end{verbatim}
  \item Return the device name (retrieved from
        org.bluez.Device.GetProperties() method which returs all the
        properties of the attached Bluetooth adapter) from a given
        Bluetooth address.
    \begin{verbatim}
      $ test-device name <Bluetooth address>
    \end{verbatim}
  \item Return the alias name (retrieved from
        org.bluez.Device.GetProperties() method which returns all the
        properties of the attached Bluetooth adapter) from a given
        Bluetooth address.
    \begin{verbatim}
      $ test-device alias <Bluetooth address> [alias]
    \end{verbatim}
  \item Set the value of the property named ``Trusted'' by passing as
        third argument either the values ``yes'' or ``no''
        (this command also requires an address to be passed as second
         argument).
    \begin{verbatim}
      $ test-device trusted <Bluetooth addres> [yes/no]
    \end{verbatim}
  \item Set the value of the property named ``Blocked'' by passing as
        third argument either the values ``yes'' or ``no''
        (this command also requires an address to be passed as second
         argument).
    \begin{verbatim}
      $ test-device blocked <Bluetooth address> [yes/no]
    \end{verbatim}
\end{enumerate}

The test-attrib is a Python script for testing purposes only and which is part
of the BlueZ toolchain. Basically, this script test the Generic Attribute
D-Bus API. Before using test-attrib the remote device needs to be paired using
test-device, simple-agent or any other client.

\subsection{Generic Attribute API}

There are basically 4 operations that you can perform with test-attrib
script:
\begin{enumerate}
   \item For each Bluetooth LE device, return its address, as well for
     each service of such Bluetooth LE device, and then return the service
     object path, UUID, and all characteristics of it, respectively.
     \begin{verbatim}
       $ test-attrib list
     \end{verbatim}
   \item Return all GATT services supported by the device.
     \begin{verbatim}
       $ test-attrib services <Bluetooth address>
     \end{verbatim}
   \item Return all discovered characteristics that belongs to a
     given service object path(primary service).
     \begin{verbatim}
       $ test-attrib discover <Service object path>
     \end{verbatim}
   \item Return all properties that belongs to a given GATT characteristic.
     \begin{verbatim}
       $ test-attrib chars <Service object path>
     \end{verbatim}
\end{enumerate}

\subsection{Managing local Bluetooth adapter}
The test-adapter is a Python script for testing purposes only and which is part
of the BlueZ toolchain. This script will basically make use of the
Adapter D-Bus API that contains all methods needed to handle Bluetooth LE
adapters.

There are basically 9 operations that you can perform with
test-adapter script:
\begin{enumerate}
  \item Retrieve Bluetooth adapter's address.
    \begin{verbatim}
      $ test-adapter address
    \end{verbatim}
  \item List all available Bluetooth adapters, as well as their object
        path, and pair of key and value for all their properties.
    \begin{verbatim}
      $ test-adapter list
    \end{verbatim}
  \item Retrieve Bluetooth adapter's name (if no argument specified),
        otherwise set the property named ``Name'' to the value passed
        as argument. Once it's got its new name set, all other
        Bluetooth devices will see this new name when looking for
        Bluetooth devices to pair with.
    \begin{verbatim}
      $ test-adapter name [name]
    \end{verbatim}
  \item Set the property named ``Powered'' to the value passed as
        argument (if any), otherwise return the value that was
        previously set in it.
    \begin{verbatim}
      $ test-adapter powered [on/off]
    \end{verbatim}
  \item Set the property named ``Pairable'' to the value passed as
        argument (if any), otherwise return the value that was
        previously set in it.
    \begin{verbatim}
      $ test-adapter pairable [on/off]
    \end{verbatim}
  \item Set the property named ``PairableTimeout'' to the value passed
        as argument (if any), otherwise return the value that was
        previously set in it. Note also that the value passed as
        argument must be an unsigned integer.
    \begin{verbatim}
      $ test-adapter pairabletimeout [timeout]
    \end{verbatim}
  \item Set the property named ``Discoverable'' to the value passed as
        argument (if any), otherwise return the value that was
        previously set in it.
    \begin{verbatim}
      $ test-adapter discoverable [on/off]
    \end{verbatim}
  \item Set the property named ``DiscoverableTimeout'' to the value
        passed as argument (if any), otherwise return the value that
        was previously set in it. Note also that the value passed as
        argument must be an unsigned integer.
    \begin{verbatim}
      $ test-adapter discoverabletimeout [timeout]
    \end{verbatim}
  \item Return the current value of the property named ``Discovering''.
    \begin{verbatim}
      $ test-adapter discovering
    \end{verbatim}
\end{enumerate}

\section{Supporting tools}

explain here l2test, gatttool, hcidump

\end{document}
