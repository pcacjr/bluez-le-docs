\documentclass[11pt]{article}
\usepackage[utf8]{inputenc}
\usepackage{graphicx}
\usepackage{fancyhdr}

\pagestyle{fancy}
\lhead{}
\chead{}
\rhead{}
\fancyfoot[C] {This document is confidential}
\fancyfoot[R] {\thepage}
\fancyhead[C] {BlueZ BTLE How-To}
\fancyhead[R] {\thepage}

\setlength{\parskip}{10pt}

\title{BlueZ BTLE How-To} \author{João Paulo Rechi Vita \\
                                  jprvita@openbossa.org}

\begin{document}

\maketitle
\newpage

\tableofcontents
\newpage

\section{Introduction}

This document describes how to test and use Bluetooth Low-Energy on Linux
systems, using the most recent implementation available at the moment of this
writing. The instructions are based on a Ubuntu Linux 11.10, since its the most
widely deployed Linux distribution nowadays. Nonetheless, the process for other
distributions should be very similar.

\section{Dependencies}

Bellow are listed the libraries and tools needed to compile BlueZ for BTLE
development. The list uses the packages name on Ubuntu 11.10.

\begin{itemize}
 \item git
 \item automake
 \item byacc
 \item flex
 \item libtool
 \item libdbus-1-dev
 \item libglib2.0-dev
 \item libcap-ng-dev
 \item libusb-dev
 \item libudev-dev
 \item libreadline-dev
\end{itemize}

To install packages on Ubuntu, use \verb|apt-get|, like shown bellow.

\begin{verbatim}
$ sudo apt-get install <package>
\end{verbatim}

Make sure you have {\em libreadline-dev} installed, otherwise \verb|gatttoll| is not
built.

\section{Getting the source code}

\subsection{Linux kernel}

BTLE development is focused on the new Management API, which requires a very
recent kernel tree. The most up-to-date kernel tree with several LE-specific
fixes can be found on the {\em integration} branch of the following tree on
GitHub. Eventually this branch will be integrated on the official kernel trees.
Don't bother trying to compile other branches, since they have experimental
changes. To access this code, clone the tree and checkout the branch:

Additionally, one should take special care when doing changes on top of this
branch, since it's going to be constantly rebased on top of the {\em
bluetooth-next} kernel tree.

\begin{verbatim}
$ git clone git://github.com/aguedes/linux.git
$ cd linux
$ git checkout integration
\end{verbatim}

\subsection{Userspace}

The upstream BlueZ tree has all the code necessary for testing and do further
BTLE development on Linux.

\begin{verbatim}
$ git clone git://git.kernel.org/pub/scm/bluetooth/bluez.git
\end{verbatim}

\section{Building and running}

\subsection{Linux kernel}

Now, copy the Linux kernel configuration file from Ubuntu as \verb|.config| to
the Linux kernel sources directory, and generate a new configuration based on
the Ubuntu one.

\begin{verbatim}
$ cd linux
$ cp /boot/config-`uname -r` .config
$ make oldconfig
\end{verbatim}

You should just press Enter for each question until this command returns. Now
you're ready to compile your kernel. To build and generate a Debian package
that can be installed on Ubuntu the packages {\em fakeroot} and {\em dpkg-dev}
are needed.  Make sure they're installed and run:

\begin{verbatim}
$ make deb-pkg
\end{verbatim}

Then install the new kernel packages with {\em dpkg}:

FIXME: howto load bluetooth module with mgmt enabled

\begin{verbatim}
$ cd ..
$ sudo dpkg -i linux-image* linux-headers*
\end{verbatim}

After the installation is complete, reboot to use the new kernel.

\subsection{Userspace}

BlueZ has a script that does the {\em bootstrap} and {\em configure} phases
with the recommended options for developers. Using it will ease the test and
development processes and is highly recommended. After downloading the source
code, run this script to prepare for compilation.

\begin{verbatim}
$ cd bluez
$ ./bootstrap-configure
\end{verbatim}

Due to a bug in Debian (thus Ubuntu) packaging, if you have the {\em check}
package installed you have to pass the link flag for the correct version of
{\em libcheck}.

\begin{verbatim}
CHECK_LIBS="-lcheck_pic" ./bootstrap-configure
\end{verbatim}

And then build the binaries with {\em make}.

\begin{verbatim}
$ make
\end{verbatim}

Now you're ready to run the Bluetooth daemon you've just compiled. First stop
the one that came with you distro:

\begin{verbatim}
$ sudo service bluetooth stop
\end{verbatim}

And the manually start it from the source tree, without forking into a daemon
(n) and with debug enabled (d).


\begin{verbatim}
$ sudo src/bluetoothd -nd
\end{verbatim}

\section{Testing Management Interface}

describe here how to pair and start discovery using btmgmt


\section{Enabling debug}

howto enable kernel debug: dmesg debug level, dynamic debug

\begin{verbatim}

modprobe bluetooth enable_le=1 enable_mgmt=1
dmesg -n 7

echo "file net/bluetooth/smp.c +pf\" >> /sys/kernel/debug/dynamic_debug/control
echo "file net/bluetooth/mgmt.c +pf\" >> /sys/kernel/debug/dynamic_debug/control
echo "file net/bluetooth/hci_sock.c +pf\" >> /sys/kernel/debug/dynamic_debug/control
echo "file net/bluetooth/hci_event.c +pf\" >> /sys/kernel/debug/dynamic_debug/control
echo "file net/bluetooth/hci_core.c +pf\" >> /sys/kernel/debug/dynamic_debug/control
echo "file net/bluetooth/hci_conn.c +pf\" >> /sys/kernel/debug/dynamic_debug/control
echo "file net/bluetooth/l2cap_sock.c +pf\" >> /sys/kernel/debug/dynamic_debug/control
echo "file net/bluetooth/l2cap_core.c +pf\" >> /sys/kernel/debug/dynamic_debug/control
\end{verbatim}

\section{Testing using python scripts}

describe here test-device, test-attrib, test-adapter and simple-agent

\section{Supporting tools}

explain here l2test, gatttool, hcidump

\end{document}
